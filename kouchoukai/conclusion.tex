\begin{frame}{Conclusion}
本研究: \underline{\textbf{同期現象を記述する数理モデルの研究}}

\begin{tcolorbox}[enhanced,
  colframe=deepskyblue,
  colback=deepskyblue!20!white,
  coltitle=black,
  drop fuzzy shadow,
  title=\textbf{結合位相振動子系}]
  \[\displaystyle \frac{\diff\theta_{i}}{\diff t}=\omega_{i}+\sum_{j\ne i}\Gamma_{ij}(\theta_{j}-\theta_{i})\]
\end{tcolorbox}

\begin{tcolorbox}[enhanced,
  colframe=forestgreen,
  colback=forestgreen!20!white,
  coltitle=black,
  drop fuzzy shadow,
  title=\textbf{理論的側面}]
\begin{itemize}
  \item \underline{\textbf{現実に即したネットワークにおける臨界指数を調べる}}\\
  $\Longrightarrow$ SWネットワークにおいて全結合と異なる臨界指数を取る
  \item \underline{\textbf{できるだけ密な同期しないネットワークを探索する}}\\
  $\Longrightarrow$ $\mu=0.6838\dots$なるネットワークを整数計画問題により発見
\end{itemize}
\end{tcolorbox}

\begin{tcolorbox}[enhanced,
  colframe=salmon,
  colback=salmon!20!white,
  coltitle=black,
  drop fuzzy shadow,
  title=\textbf{実験的側面}]
\begin{itemize}
  \item \underline{\textbf{ガウス過程回帰を用いて結合関数を推定できるか?}}\\
  $\Longrightarrow$ additiveな周期カーネルの提案
\end{itemize}
\end{tcolorbox}
\end{frame}